\documentclass[10pt]{article}

\usepackage{titlesec}
\usepackage{titling}
\usepackage{tabularx}
\usepackage{float}
\usepackage{polyglossia}

\usepackage[margin=1in]{geometry}

\usepackage{hyperref}

\author{Igor Motyka}
\title{Resume}
 
 \renewcommand{\maketitle}{
 \begin{center}
    \huge\bfseries\theauthor
 \end{center}
 \begin{center}
    \href{mailto:igor\_motyka@mail.com}{igor\_motyka@mail.com} --- 519 803 704 \\
 \end{center}
 \begin{center}
	 \href{https://github.com/Anav0}{Github} \hspace{0.7cm} \href{https://www.linkedin.com/in/igor-m-873439168/}{Linkedin} \hspace{0.7cm} \href{https://igormotyka.netlify.app/}{My website}
 \end{center}
 }
\renewcommand{\familydefault}{\rmdefault}
\DeclareTextFontCommand{\emph}{\bfseries\em}
 
\titleformat{\section}
{\huge\bfseries}{}{0em}{}[\titlerule]

\titleformat{\subsection}
{\Large\bfseries}{}{0em}{}

\titleformat{\subsubsection}
{\large\bfseries}{$\bullet$}{1em}{}

\renewcommand{\arraystretch}{0}

\begin{document}

\maketitle

\section{Wykształcenie}
\subsection{Uniwersytet Opolski}
\subsubsection{Magister informatyki 2020 - 2021}
\begin{table}[H]
    \begin{tabularx}{\textwidth}{@{}l X}
     \textbf{Tytuł pracy:} & \textit{``Projekt i Implementacja algorytmu symulowanego wyżarzania w celu wyznaczenia terminów rehabilitacji``} \\
    \end{tabularx}
\end{table}
\noindent W swojej pracy starałem się rozwiązać problem harmonogramowania zabiegów rehabilitacyjnych. W tym celu stworzyłem framework zdolny do rozwiązania dowolnego problemu optymalizacji za pomocą wbudowanych algorytmów metaheurystycznych. Framework został napisany w \emph{.Net 5.0} lecz analiza danych została przeprowadzona w \emph{python, pandas i matplotlib}.
\subsubsection{Inżynier informatyki 2016 - 2020}
\begin{table}[H]
    \begin{tabularx}{\textwidth}{@{}l X}
    \textbf{Tytuł pracy:} &  \textit{``Przykład komunikatora wzbogaconego o implementację czatbota``}\\
    \end{tabularx}
\end{table}
\noindent Jako projekt inżynierski stworzyłem \href{https://www.behance.net/gallery/91600605/Gymba-chat}{komunikator} napisany we \emph{Vue}, \emph{Node js}, \emph{socket.io} i \emph{MongoDb}.
\section{Kariera}
\subsection{Medinet List. 2020 - List. 2021 (12 ms)}
W firmie Medinet pracowałem z .Net 5.0 w celu optymalizacji procesu harmonogramowania zagiebów rehabilitacyjnych.
\subsection{BCF Software Lipiec 2019 - Grudzień 2019 (7 ms) }
W firmie BCF Softwere procowałem jako frontend developer z React'em, Vue oraz Nuxt'em. Oto wybrane projekty które współtworzyłem:
    \href{https://cryptogamble.com}{cryptogamble.com}, \href{https://lotterija.com}{lotterija.com}, \href{https://bgs.bet}{bgs.bet}.
\section{Certyfikaty}
\Large \textbf{Software Development Fundamentals} (Lipiec 2019) \vspace{0.3em}\\
\large Microsoft \vspace{0.5em} \\
\small ID dPYP-DTRs \\
\newpage
\section{Umiejętności}
\subsection{Programowanie}
\renewcommand{\arraystretch}{2.5}
\begin{table}[H]
    \begin{tabularx}{\textwidth}{@{}l X}
         \textbf{C\#} & Potrafię pisać kod uruchamiany wielowątkowo, Wiem jak wykorzystać techniki memoizacji. Dictionary oraz StringBuilder to moi kumple. Znam większość wzorców projektowych z książki: \em ``Head first: design patterns``. Wykorzystałem EF core w kilku swoich projektach.\\
         \textbf{.Net} & Potrafię stworzyć paczki NuGet, REST API z SQL lub Mongo jako system bazodanowy lub aplikację WPF jeśli zajdzie taka potrzeba. Wiem na jakiej zasadzie działają mikroserwisy ale nigdy żadnego nie tworzyłem. \\
         \textbf{Moq} & Potrafię napisać zmockowane testy jednostkowe lecz TDD nie jest tak super jak Bob chciałby aby było.  \\
         \textbf{MS SQL} & Potrafię napisać kwerendy bez tworzenia indeksu dla każdej kolumny oraz przy minimalnej ilości join'ów. \\
	 \textbf{Javascript / Typescript} & Znam ES6, event loop oraz co to jest prototype.\\
	 \textbf{Node.js} & Potrafię stworzyć typowy backend wraz z REST API.\\
         \textbf{MongoDb} & Wykorzystałem mongo w pracy inżynierskiej razem z \emph{Mongoose ORM}. \\
         \textbf{Frontend} & Znam każdy z nich w różnym stopniu wtajemniczenia. Tutaj są one wypisane od najlepiej opanowanego do najmniej: \bfseries \em Svelte, React, Gatsby, Nuxt, Vue. \\
         \textbf{Gsap} & Potrafię animować pliki svg.\\
         \textbf{Python} & Potrafię stworzyć proste Flask API, skrypty np. do parsowanie stron internetowych (scraping) czy implementacji algorytmu np. Moore'a, Hoopcroft'a. Oprócz tego potrafię stworzyć ładne wykresy (\bfseries \em matplotlib, pandas, seaborn, numpy). \\
         \textbf{Rust} &Dopiero się go uczę lecz muszę powiedzieć, że jest to najlepiej udokumentowany język z jakim pracowałem. Obecnie tworze dwa programy aby w pełni zrozumieć ten język (epub reader, linux bootstraper). \\
    \end{tabularx}
\end{table}
\subsection{Inne}
\begin{table}[H]
        \begin{tabularx}{\textwidth}{@{}l X}
         \LaTeX & Potrafię stworzyć eleganckie dokumenty. \\
         \textbf{Postman} & W celu dokumentacji i testowania API.  \\
         \textbf{Jira} & Do zarządzania zadaniami.  \\
\end{tabularx}
\end{table}
\section{Języki}
\begin{table}[H]
        \begin{tabularx}{\textwidth}{@{}l X}
         \bfseries Angielski & C1. \\
         \bfseries Polski & Ojczysty.  \\
\end{tabularx}
\end{table}
\section{Zainteresowania}
\begin{itemize}
    \item \href{http://cejsh.icm.edu.pl/cejsh/element/bwmeta1.element.desklight-b5b6fec4-8161-42ac-9a31-c0e6c344f9fa}{Sunnicka i szyicka propaganda w poezji śpiewanej}.
    \item Historyczne gry wojenne (Kriegspiel).
    \item Podziwianie komunistycznej architektury z grzbietu mojego roweru.
\end{itemize}
\vfill
\begin{center}
Zezwalam każdemu kto jest w posiadaniu tego dokumentu do wykorzystania go w celach rekrutacyjnych.
\end{center}
\end{document}

